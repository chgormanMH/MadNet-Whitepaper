\subsection{Hash-to-curve Functions}
\label{ssec:hash-to-curve}

The contents of this section should not be considered a full proof
of security but are for implementation reference.
Formal treatment of the security proofs for hash-to-curve
functions are handled in referenced works.

We turn our attention to the hash-to-curve functions
implemented in our system.
The contents of this section are described hereafter.
First, we review a mechanism often employed to solve this problem.
Then, we briefly discuss why this solution is not ideal in our setting.
Finally, we introduce the basis of our implementation and proceed into the
definition of the algorithms used in our system after a brief review of
some required mathematical operations.
All algorithms may be found at the end of this section.

The seemingly standard, non-deterministic method to perform a hash-to-curve
operation in the elliptic curve setting is to use the ``MapToGroup''
method as presented in the original BLS short signature
paper~\cite{boneh2001short}.
In this method values are hashed into
$\F_{p}$ by modular arithmetic with a concatenated counter.
This counter starts at a fixed value, and is incremented until
a value is found that creates a valid point on the specified
elliptic curve.
While this may be sufficient in some instances, we prefer 
deterministic methods due to the need for bounded
computational overhead.
In our system, the need for bounded computational
overhead arises from a desire to allow an Ethereum smart contract to
perform the hash-to-curve operations with bounded gas consumption.
Deterministic methods also have the benefit of minimizing
side-channel attacks in those
algorithms that require such protection.

The hash-to-curve implementation selected allows the problem space to be
divided into independent problems such that their solutions may be composed.
Specifically, we first hash to the base field
(hash map $\mathfrak{h}:\braces{0,1}^{*}\to \F$)
and then find a deterministic map from the base field
to the elliptic curve (function $f:\F\to E(\F)$).
This approach allows for a separation of concerns and has become a
standard approach to the problem of hashing into an elliptic curve
~\cite{icart2009hash,ft2012bnhashtocurve,boneh2019h2cBLS12}.
Although this strategy does offer a simplified view of the problem,
mapping from the base field to the elliptic curve is nontrivial.
Additionally, it is frequently the case that $f$ is not surjective, but
we can overcome this limitation to obtain a surjective hash-to-curve algorithm
under easily-satisfied
conditions~\cite{tibouchi2014elligator,ft2012bnhashtocurve}.
Specifically, we use domain-separation in order to obtain independent hash
functions.
These independent hash functions allow us overcome a non-surjective
mapping.
We will fully address this concern later in this document.

Our discussion of hashing functions follow
\cite{ft2012bnhashtocurve,boneh2019h2cBLS12}.
We ask the reader to note that in the remaining work of this section,
we view the group $E(\F_{p})$ additively unless otherwise specified.
We highlight this fact because this is different from the multiplicative
notation used previously in this document.
Before we define our implementation of the hash-to-curve algorithms employed,
we feel a review of the mathematics would benefit the reader.
Thus, we first
review the mathematics necessary to perform the hash-to-curve operations.
After this review, we present the algorithm for hashing to $\G_{1}$
and then present the algorithm for hashing to $\G_{2}$.

For reference, the full \textsc{HashToG1} algorithm can be
found in Alg.~\ref{alg:hash-to-G1} and the full algorithm for \textsc{HashToG2}
can be found in Alg.~\ref{alg:hash-to-G2}.



\subsubsection{Inverses, Square Roots, and Legendre Symbols
    in $\F_{p}$}
\label{sssec:finite_math_Fp}

In this section we review some finite field mathematics that are
important in our hash-to-curve setting.

We begin by reviewing inversion in $\F_{p}$.
First, we recall $\F_{p}^{*} \equiv \F_{p}\setminus\braces{0}$,
the nonzero elements of our finite field.
For $a\in\F_{p}^{*}$, we have Euler's formula:

\begin{equation}
    a^{p-1} = 1 \mod p.
\end{equation}

\noindent
This implies $a^{-1} \mod p = a^{p-2}$.
We acknowledge more efficient methods for computing modular inverses are
possible, but exponentiation can easily be performed in constant time, which
is a goal of our implementation. This concludes our discussion of computing
inverses in $\F_{p}$.

We now review the mechanisms for
computing Legendre symbols.
We recall that $a\in\F_{p}^{*}$ is a \emph{quadratic residue}
if there is $x\in\F_{p}$ such that $x^{2} = a \mod p$;
otherwise, $a$ is a called a \emph{quadratic nonresidue}.
This allows us to define the Legendre symbol~\cite{imc2008}:

\begin{equation}
    \chi_{p}(a) \equiv \begin{cases}
        1 \quad& \text{if there $a$ is a quadratic residue modulo $p$.} \\
        -1 \quad& \text{if there $a$ is a quadratic nonresidue modulo $p$.} \\
        0 \quad& \text{if $p \mid a$}
    \end{cases}
\end{equation}

\noindent
The Legendre symbol has a simple formula:

\begin{equation}
    \chi_{p}(a) = a^{\frac{p-1}{2}}.
\end{equation}

\noindent
This formula holds even when $a=0$.
This concludes our discussion of computing the Legendre Symbol in $\F_{p}$.

Computing square roots is more challenging.
We now review the mechanism of computing square roots.
In our case $p = 3\mod 4$, so there is a simple
formula to solve $x^{2} = a$:

\begin{equation}
    x = a^{\frac{p+1}{4}}.
\end{equation}

\noindent
This relies on the assumption that $a$ is a quadratic residue,
so $a^{\frac{p-1}{2}} = 1$.
This concludes our discussion of computing the square roots in $\F_{p}$.

Taking inverses, Legendre symbols, and square roots
gives us the necessary tools to hash to $\G_{1}$.
This ends the preliminary work necessary to develop
\textsc{HashToG1}.


\subsubsection{Inverses, Square Roots, and Legendre Symbols
    in $\F_{p^{2}}$}
\label{sssec:finite_math_Fp2}

The previous section reviewed the mathematics of inverses, square roots, and
Legendre Symbols in $\F_{p}$.
We must be able to perform the same operations
in $\F_{p^{2}}$ if we wish to hash to $\G_{2}$.
While computing square roots and
inverses may be familiar to the reader in $\F_{p}$, the operations require
special handling in $\F_{p^{2}}$.
We review these mechanisms now.

We first address the problem of computing inverses in $\F_{p^{2}}$.
Our discussion and methods follow~\cite{adj2012sqrtEvenExt},
and we present their general results applied to our particular case.
As mentioned above, the setting of our elliptic curve is $p = 3\mod 4$.
This implies $-1$ is a quadratic nonresidue.
Thus, we have the following isomorphism:

\begin{equation}
    \F_{p^{2}} \simeq \F_{p}[i]/\parens{i^{2}+1}.
\end{equation}

\noindent
This is the analogous to constructing the complex numbers $\C$ from
the real numbers $\R$.
Our implementation uses this construction. We may use
this construction to compute an efficient inversion as follows:

\begin{equation}
    \parens{a_{0} + a_{1}i}^{-1} = \frac{a_{0}-a_{1}i}{a_{0}^{2}+a_{1}^{2}}.
\end{equation}

\noindent
The main computational cost of this operation is the inversion of an element
in $\F_{p}$.
This concludes our discussion of computing inverses in $\F_{p^{2}}$.

We now begin our discussion of computing the Legendre Symbol in $\F_{p^{2}}$
If $a = a_{0} + a_{1}i\in\F_{p^{2}}$, then $a$ is a
quadratic residue in $\F_{p^{2}}$ if and only if $a_{0}^{2} + a_{1}^{2}$
is a quadratic residue in $\F_{p}$.
The main computational cost of this operation is the computation of the Legendre
symbol of an element in $F_{p}$.
From the above we have the Legendre symbol in $\F_{p^{2}}$,
which we denote as \textsc{LegendreFP2} or $\chi_{p^{2}}(\cdot)$;
this particular algorithm is presented in
Alg.~\ref{alg:legendre-fp2}.
This concludes our discussion of computing the Legendre Symbol in
$\F_{p^{2}}$.

We will now look at computing square roots in $\F_{p^{2}}$.
The main idea is to find $b\in\F_{p^{2}}$ and odd $s$ such
$b^{2}a^{s} = 1$.
In this case, we see

\begin{align}
    \brackets{ba^{\frac{s+1}{2}}}^{2} &= b^{2}a^{s+1} \nonumber\\
        &= a.
\end{align}

\noindent
If we set

\begin{align}
    b &= \parens{1 + a^{\frac{p-1}{2}}}^{\frac{p-1}{2}} \nonumber\\
    s &= \frac{p-1}{2},
\end{align}

\noindent
then when $b\ne0$, we have $b^{2}a^{s} = 1$.
When $b=0$, we have $a^{\frac{p-1}{2}}=-1$.
In this case, our square root is $ia^{\frac{p+1}{4}}$.
This procedure is formally presented in~Alg.~\ref{alg:sqrt-fp2}.
The main computational cost of this operation is two exponentiations
in $\F_{p^{2}}$.
This concludes our discussion of computing square roots in $\F_{p^{2}}$
and the preliminary review of those operations
necessary to develop \textsc{HashToG2}.


\subsubsection{Hashing to Base}
\label{sssec:hash-to-base}

In this section we will describe the hash-to-base operation.
We begin by discussing the construction of a random oracle into
$\F_{p}$ using a single cryptographic hash function with
domain separation.
We then bound the deviation from uniformity in this operation.

In the following $H$ is a $256$-bit hash function.
In our implementation we use \textsc{Keccak256}.
Note this is the \textsc{Sha3} variant used by Ethereum
that differs from the NIST approved \textsc{Sha3} hash function due to a
difference in the handling of padding.

Let $H$ act as a random oracle.
We map from $H$ to $\Z$ by interpreting the output of a
cryptographic hash function as a big endian unsigned integer.
This can be seen in lines 4 and 5 of $\textsc{HashToBase}$
as the function named $\texttt{b2u}$.
Although we may naively map from $\Z$ to $\Z_{p}$ by simply taking the
output of $H$ modulo $p$, this would be insecure in our setting due to the
nonuniformity of this operation.
In order to ensure the mapping from $\Z$ to $\Z_{p}$ is more
uniformly distributed, domain separation is utilized.
Specifically, we use domain separation in order obtain independent hash
functions from $H$ through concatenation of constants with the message being
hashed.
These independent hash functions allow us to create a secure 512-bit random
number from a single cryptographic hash function.
The full explanation of this operation is below.

Let $\textsc{HashToBase}:\braces{0,1}^{*}\times\braces{0,1}^{8}\times
\braces{0,1}^{8}\to\F_{p}$ denote our random oracle into the
underlying field $\F_{p}$.
In the $\textsc{HashToBase}$ operation the first component corresponds to the
underlying message being hashed, while the last two elements provide the
necessary domain separation.
See Alg.~\ref{alg:hash-to-fp} for the full implementation.

Due to the fact $p$ is prime and not a power of 2, there will be some
nonuniformity in the resulting distribution of \textsc{HashToBase}.
We investigate this nonuniformity now.

First, we assume that $H:\braces{0,1}^{*}\to\Z_{N}$
is a random oracle and $p\in\braces{1,2,\cdots,N-1}$.
We want to determine how much $H(m)\mod p$ deviates from uniformity.
We restrict ourselves to the case when $p\nmid N$.
Let

\begin{equation}
    N = qp + r
\end{equation}

\noindent
with $0\le r< p$ and $q\ge1$.
Because $p\nmid N$, we have $r\ge1$.
Let $X$ be uniformly distributed on $\Z_{N}$ and set $X_{p} = X \mod p$.
Furthermore, we let $U_{p}$ be the uniform distribution modulo $p$.
Then

\begin{equation}
    \mathcal{P}\parens{X_{p}\in\braces{0,\cdots,r-1}} = \frac{q+1}{N}
\end{equation}

\noindent
and

\begin{equation}
    \mathcal{P}\parens{X_{p}\in\braces{r,\cdots,p-1}} = \frac{q}{N}.
\end{equation}

\noindent
Here, $\mathcal{P}$ denotes the probability of an event occurring.
We now determine the deviation of $X_{p}$ from the uniform
distribution $U_{p}$:

\begin{align}
    \Delta\parens{X_{p},U_{p}} &\equiv
        \sum_{k=0}^{p-1} \abs{\mathcal{P}(X_{p}=k) - \mathcal{P}(U_{p}=k)}
        \nonumber\\
    &= \sum_{u=0}^{r-1} \abs{\frac{q+1}{N} - \frac{1}{p}} +
        \sum_{u=r}^{p-1} \abs{\frac{q}{N} - \frac{1}{p}} \nonumber\\
    &= r\frac{qp + p - N}{pN} + \parens{p-r}\frac{N-qp}{pN}
        \nonumber\\
    &\le \frac{p}{N}.
\end{align}

\noindent
From this, we see that if $p$ is a $k$-bit prime and
$N = 2^{k+\ell}$, then the deviation from uniformity is $\le2^{-\ell}$.

In our case, $p$ is a 254-bit prime and we concatenate the output
of independent hash functions in order to have a uniformly distributed
512-bit output.
From the above it may be seen that \textsc{HashToBase} produces outputs whose
deviation from uniformity is less than $2^{-258}$.
This deviation is sufficiently small as to not be of concern.
Further work is required to more formally prove this assumption, but
this work is not included at this time.


\subsubsection{Base to $\G_{1}$}
\label{sssec:base-to-G1}

In this section we discuss the construction of a hash-to-G1 algorithm.
We begin by noting the non-surjective nature for many functions
$f:\F_{p}\to E(\F_{p})$.
Then, we cite a known solution to this problem and provide necessary
mathematics to understand the fundamental operation that
overcomes the problem.
Next, we discuss the actual implementation.
Lastly, we discuss specific exclusion of three points from the allowable
outputs of this algorithm for security purposes.

Let us now suppose that we have a hash function
$\mathfrak{h}:\braces{0,1}^{*}\to\F_{p}$ and a
deterministic map $f:\F_{p}\to E(\F_{p})$.
As has been previously stated, in many
situations~\cite{icart2009hash,ft2012bnhashtocurve,boneh2019h2cBLS12}
$f$ is not surjective.
That is, there are points on the elliptic curve $E(\F_{p})$
(sometimes a nontrivial fraction) which cannot be reached by $f$.
Even so, if we use domain separation to construct independent hash functions,
$\mathfrak{h}_{1}, \mathfrak{h}_{2}:\braces{0,1}^{*}\to\F$,
then

\begin{equation}
    F(\texttt{m}) = f(\mathfrak{h}_{1}(\texttt{m}))
                    + f(\mathfrak{h}_{2}(\texttt{m}))
\end{equation}

\noindent
is indistinguishable from a random oracle on $E(\F_{p})$
under certain restrictions on $f$.
See~\cite{ft2012bnhashtocurve,tibouchi2014elligator} for details.

We now turn our attention to determining $f:\F_{p}\to E(\F_{p})$.
Finding a map $f:\F_{p}\to E(\F_{p})$ is involved.
Our BN curve has the form

\begin{align}
    E:y^{2} &= g(x) \nonumber\\
            &= x^{3} + b.
\end{align}

\noindent
From~\cite{ft2012bnhashtocurve}, it possible to show there
are $x_{1},x_{2},x_{3},y\in\F_{p}$ such that

\begin{equation}
    g(x_{1})g(x_{2})g(x_{3}) = y^{2}.
\end{equation}

\noindent
When $y\ne0$, quadratic reciprocity implies that $g(x_{i})$ is square
for some $i$; that is, for some $i$ we have
$(x_{i},\sqrt{g(x_{i})})\in E(\F_{p})$.
For uniqueness, we choose the smallest $i$.
We will use the construction of~\cite{ft2012bnhashtocurve}
to determine such points with some modifications based
on work in~\cite{boneh2019h2cBLS12}.
The exact algorithm can be found in Alg.~\ref{alg:base-to-G1}.

One of the main difficulties is determining $i$ in such a way
as to not leak information; because of this, we do not wish to
rely upon if statements.
In~\cite{ft2012bnhashtocurve}, they suggested the function

\begin{equation}
    \psi(r_{1},r_{2}) = \brackets{\parens{r_{1}-1}r_{2}\mod3} + 1.
    \label{eq:ftAB}
\end{equation}

\noindent
This function works under the assumption that modular
arithmetic always returns positive integers.
This is not always the case in programming languages;
in particular, it does not hold in \textsc{Go} (Golang), the language
we use to implement these algorithms.
To circumvent this, we use the following function,
which is implemented in Line~\ref{alg_line:new_coef_func}
of Alg.~\ref{alg:base-to-G1}:

\begin{equation}
    \parens{r_{1}-1}\parens{r_{2}-3}/4 + 1.
    \label{eq:newAB}
\end{equation}

In~\cite{ft2012bnhashtocurve}, the authors encountered an issue when $t=0$.
Their solution was to define

\begin{equation}
    \textsc{BaseToG1}(0) = \parens{\frac{-1+\sqrt{-3}}{2}, \sqrt{1+b}}.
    \label{eq:baseToG1zero}
\end{equation}

\noindent
Wahby and Boneh~\cite{boneh2019h2cBLS12} suggest another
method, which we implement, in order to have a more efficient
algorithm.
This implementation also affords the benefit of not needing to
handle the case of $t=0$ separately.
This can be seen in Line~\ref{alg_line:alpha_def}
of Alg.~\ref{alg:base-to-G1} where we define $\alpha$ as the inverse
of a value which depends on $t\in\F_{p}$.
When $t=0$, we compute $\alpha = 0$
because we compute inversions via exponentiation.
Thus, the computational convention $0^{-1} = 0 \mod p$ is established.
This leads to the same result as in Eq.~\eqref{eq:baseToG1zero}
without special handling.

We have cryptographic insecurity when
$\textsc{HashToG1}(\texttt{m}) = g_{1}^{\alpha}$ for known $\alpha$.
Note we briefly switch back to multiplicative notation at this time.
In practice, it is likely this insecurity will only be known when
$\alpha\in\braces{-1,0,1}$ or is sufficiently close to these values.
We assume we may not fix simple proximity to these values, and thus only
explicitly address the case of $\alpha\in\braces{-1,0,1}$.
For clarity, this notation specifies the hash function outputs of
the identity element, the generator, or the generator's
inverse.
Due to the concerns around the use of these points, we will not allow these
individual cases.
In order to enforce this requirement we include a
\textsc{SafePointCheck} in \textsc{HashToG1};
see Line~\ref{alg_line:safe_point_check} in Alg.~\ref{alg:hash-to-G1}.
This function checks the point returned from the hash function for
equivalence with the identity element OR equivalence of the point's
$x$ coordinate with $1$.
If either of these conditions is true, an error is raised.
Although the probability of a hash mapping to these points is small, this
error must be handled appropriately to prevent an attacker from causing
unexpected errors in a remote system due to specially crafted messages.


\subsubsection{Base to $\G_{2}$}
\label{sssec:base-to-G2}

We now take up the slightly more challenging possibility
of computing a hash function to $\G_{2}$.
Fortunately, we may utilize many of
the previously described operations.
Thus, we do not repeat those explanations and only discuss
those operations that differ.
We would like to remind the reader that the results in
Sec.~\ref{sssec:finite_math_Fp2} allow
us to compute Legendre symbols and square roots in $\F_{p^{2}}$.

Using the derivation of~\cite[Section 3]{boneh2019h2cBLS12},
we set $u_{0}=1$ as in $\G_{1}$ to obtain

\begin{align}
    x_{1} &= \frac{\sqrt{-3}-1}{2} - \frac{t^{2}\sqrt{-3}}{t^{2}+g'(1)}
        \nonumber\\
    x_{2} &= -1 - x_{2} \nonumber\\
    x_{3} &= 1 - \frac{\parens{t^{2}+g'(1)}^{2}}{3t^{2}},
\end{align}

\noindent
for $t\in\F_{p^{2}}$.
Here, we have

\begin{equation}
    E':y^{2} = g'(x) = x^{3} + b',
\end{equation}

\noindent
where $b' = b/\xi = 3/\parens{i+9}$.
With this choice, $-g'(1)$ is a residue in $\F_{p^{2}}$.
This ensures that $t\in\{0,\pm\sqrt{-g'(1)}\}$
implies $x_{1}$ is a valid point on the curve;
thus, all inputs result in valid outputs.
See Alg.~\ref{alg:base-to-twist} for the algorithm.
Thus, our hash functions to $\G_{1}$ and $\G_{2}$ are
essentially the same.

At this point, we have successfully mapped into $E'$;
however, our goal is to map into $\G_{2}$.
From~\cite{bnCurves}, we know $\abs{E'(\F_{p^{2}})} = q\parens{2p-q}$.
This gives a cofactor $r = 2p-q$ because $p\nmid q$.
We take care of this by clearing the cofactor.
See Alg.~\ref{alg:hash-to-G2} for the full hash algorithm.

\clearpage

\input{algs/hash_to_G1.tex}

\input{algs/gfp2_algs.tex}

\input{algs/hash_to_G2.tex}
