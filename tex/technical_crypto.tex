\section{Technical Cryptography}
\label{sec:tech_crypto}

In this section we present in detail the specifics of the
cryptography we will be using for MadNetwork.
At times we will be verbose in our algorithmic details
and design choices to allow for others to understand our decisions.

We will begin by comparing this work with the
Ethereum Distributed Key Generation whitepaper~\cite{ethdkg}
in Sec.~\ref{ssec:ethdkg_comparison};
our design is based this paper.
Some of the implementation-specific details are covered in
Sec.~\ref{ssec:pk_curve_specifics}.
In Sec.~\ref{ssec:math_def}, we discuss the mathematics
related to pairing-based cryptography; this is integral to
our work as it is required for our group signatures.
We describe the distributed key generation protocol in
Sec.~\ref{ssec:dkg};
the specific method of shared secret encryption is
described in Sec.~\ref{ssec:secret_enc}.
We discuss how to construct group signatures in
Sec.~\ref{ssec:crypto_group_sig}.
Group signatures from pairing-based cryptography require
a hash-to-curve function, and we talk about our construction
based on~\cite{ft2012bnhashtocurve,boneh2019h2cBLS12}
in Sec.~\ref{ssec:hash-to-curve}.

We follow~\cite{ethdkg} in our definition of
$\parens{t,n}$-threshold system, where we need $t+1$ actors
for consensus.
Unfortunately, this is different than what was used previously,
where $\parens{t,n}$-threshold system meant $t$ actors were
needed to agree.
We keep this difference for ease of comparison with the referenced paper.

\subsection{Comparison with Ethereum Distributed Key Generation Paper}
\label{ssec:ethdkg_comparison}

In~\cite{ethdkg}, the authors presented the Ethereum
Distributed Key Generation whitepaper.
Here, participants work together to form a master public key
(the public key for the group)
based on verifiable secret sharing.
The master secret key (the private key for the group)
is the summation of the shared secrets correctly shared by
validators.
This system may proceed even with a limited number of Byzantine actors.
In the end, participants may compute partial signatures of a message
which may be combined to form a valid group signature.

While we follow the procedure of~\cite{ethdkg} in general,
there are other concerns that must be taken into consideration.
The MadNetwork will be a sidechain of Ethereum,
and the keys we construct will be used to sign the blocks of our sidechain.
By anchoring into Ethereum, we are able to use smart contracts to enforce
compliance with the consensus algorithm as well as punish
those who behave in a cryptographically-verifiable malicious way.
Malicious behavior include submitting false keys, submitting
false proofs, signing invalid messages, and similar actions.
This is different from the original paper where everything
happened on Ethereum and participants who
acted maliciously during the key derivation stage would still
be allowed to proceed because honest actors would be able to
work together to derive the correct information.
In that setting, the focus was having a $\parens{t,n}$-threshold
system whereby $t+1$ actors are required to work together to
sign messages for the group.
Here, $t$ and $n$ are preassigned.
In our case, we specifically desire a Byzantine-fault tolerant
system, whereby we require $t = \ceil{2n/3}-1$.
Even though we have the same $t$ values for $n = 3k$ and $n = 3k-1$
(thereby potentially allowing one malicious validator to be
ejected without forcing a required restart),
we will restart the DKG process whenever malicious behavior
is cryptographically proven.
By forcing a restart upon pernicious action,
validators are discouraged from malicious activity as well
as lose the opportunity to earn block rewards.
We also restart when validators fail to submit the required
information.
In the case when malicious intent cannot be proven due to
the possibility of technical failure, a minor fine will be
given due to time cost of the other participants.

During the DKG process, all of the material necessary
to recover a participant's secret is available provided
enough actors work together.
As mentioned in~\cite{ethdkg}, this is useful in order to allow
for the DKG to continue even if certain participant's fail to cooperate,
but the problem is that with this secret information it would
be possible for a large enough malicious subset (specifically,
a majority greater than two-thirds) to recover a secret
and produce valid signatures from participant $P_{i}$
proving malicious behavior that is not, in fact, perpetrated by $P_{i}$.
This is of serious concern because participants
stake tokens on the basis of being validators
on our blockchain and receiving block rewards for their computation
and are threatened with losing stake should they
behave nefariously.

Requiring at least 67\% percent of the validators to work together
in order to produce stake-burning results is
better than a 51\% Attack which can occur
in Proof-of-Work blockchains but it leaves something to be desired;
we would like for honest validators to be immune to
the previously-mentioned behavior even if there is only
one honest participant.
To combat this, we will require all messages to be signed
by a participant's Ethereum's private key.
This will allow all messages to be validated against the Ethereum
public key and will be safe so long as the Ethereum private key
is secure.
In this way, any secret information leaked during the distributed
key generation will not enable Byzantine actors to produce
cryptographic proof of malicious behavior against honest participants.


\input{tex/tcrypt_spec_imp.tex}
\input{tex/tcrypt_math_def.tex}
\input{tex/tcrypt_dkg.tex}
\input{tex/tcrypt_sse.tex}
\subsection{Group Signatures}
\label{ssec:crypto_group_sig}

\subsubsection{Constructing the Master Public Key}

The goal of our consensus algorithm is to enable a Byzantine-fault
tolerant subgroup to cryptographically sign on behalf of the entire
group without requiring every individual group member to sign.
This is enabled by signature aggregation in the appropriate way.

We let $\mathcal{Q}$ be the collection of qualified
actors who correctly shared their secrets and
$\mathcal{R}\subseteq\mathcal{Q}$ such that
$\abs{\mathcal{R}} = t+1$; thus, $\mathcal{R}$ is a
Byzantine-fault tolerant subgroup.
As discussed in~\cite{gennaro3revisiting,gennaro1999secure,ethdkg},
in order to ensure that no bad actors gain any information
about the master public key and not be able to change
its underlying probability distribution, we require
$h_{1}\in\G_{1}$ such that $\dlog_{g_{1}}h_{1}$ is unknown.
We also let $h_{2}\in\G_{2}$ be a generator.

The individual shared secrets $s_{i}$ allow us to define the
\emph{master secret key} $\text{msk}$:

\begin{equation}
    \text{msk} = \sum_{P_{i}\in\mathcal{Q}} s_{i}.
\end{equation}

\noindent
This gives us the \emph{master public key} $\text{mpk}$:

\begin{align}
    \text{mpk} &= h_{2}^{\text{msk}} \nonumber\\
        &= \prod_{P_{i}\in\mathcal{Q}} h_{2}^{s_{i}}.
\end{align}

\noindent
Because everyone in $\mathcal{Q}$ correctly shared his secret,
a Byzantine-fault tolerant subgroup $\mathcal{R}$ can
correctly obtain the secret $s_{i}$ via Lagrange interpolation:

\begin{align}
    s_{i} &= \sum_{P_{j}\in\mathcal{R}} s_{i\to j} R_{j} \nonumber\\
    R_{j} &= \prod_{\substack{P_{k}\in\mathcal{R} \\ k\ne j}} \frac{k}{k-j}.
    \label{eq:Rj_coefs}
\end{align}

\noindent
This would allow us to recover the secret $s_{i}$ should $P_{i}$
fail to share $h_{1}^{s_{i}}$ below; however,
we take a stricter response and would view failure to share
as malicious activity leading to stake slashing.

We now proceed to compute $\text{mpk}$.
Let

\begin{equation}
    \pi(h_{1}^{s_{i}}) = \textsc{DLEQ}(
        g_{1},g_{1}^{s_{i}},h_{1},h_{1}^{s_{i}},s_{i})
\end{equation}

\noindent
be the zk-proof that $h_{1}^{s_{i}}$ is $P_{i}$'s portion
of the master public key (technically part of $\text{mpk}^{*}$
as defined below).
Because $C_{i0} = g_{1}^{s_{i}}$ is public knowledge
and $P_{i}$ correctly shared his secret $s_{i}$, it is possible
to publicly verify $h_{1}^{s_{i}}$.
Additionally, $P_{i}$ will publish $h_{2}^{s_{i}}$ so that
we can ensure

\begin{equation}
    \textsc{PairingCheck}(h_{1}^{s_{i}},\bar{h}_{2},h_{1},h_{2}^{s_{i}})
        = 1.
\end{equation}

\noindent
This will be called by a smart contract.
Thus, failure of $P_{i}$ to publish $h_{1}^{s_{i}}$,
a valid proof $\pi(h_{1}^{s_{i}})$, and the corresponding
$h_{2}^{s_{i}}$ amounts to misbehavior which will
lead to a fine.

The Ethereum smart contract will store $h_{1}^{s_{i}}$ from
all participants and broadcast
$h_{1}^{s_{i}}$, $\pi(h_{1}^{s_{i}})$, and $h_{2}^{s_{i}}$.
From here, any participant will be able to submit

\begin{equation}
    \text{mpk} = \prod_{P_{i}\in\mathcal{Q}} h_{2}^{s_{i}}
\end{equation}

\noindent
to the smart contract.
Because $\braces{h_{1}^{s_{i}}}_{i\in\mathcal{Q}}$ are stored
and valid, the contract can construct

\begin{equation}
    \text{mpk}^{*} = \prod_{P_{i}\in\mathcal{Q}} h_{1}^{s_{i}}
    \label{eq:mpk_dual}
\end{equation}

\noindent
and call

\begin{equation}
    \textsc{PairingCheck}(\text{mpk}^{*},\bar{h}_{2},h_{1},\text{mpk})
\end{equation}

\noindent
to ensure $\text{mpk}$ is valid.
The master public key can then be stored publicly and used
for group signature verification.



\subsubsection{Constructing Group Signatures}

At this point, we have successfully constructed the master public
key $\text{mpk}$ for $\mathcal{Q}$ and distributed
the master secret key $\text{msk}$ among the members of $\mathcal{Q}$.
We now turn our attention to constructing group
signatures from partial signatures.

Each participant $P_{j}\in\mathcal{Q}$ has a portion of the
master secret key; this is portion is called the \emph{group secret
key}:

\begin{equation}
    \gsk_{j} = \sum_{P_{i}\in\mathcal{Q}} s_{i\to j}.
\end{equation}

\noindent
This is possible because we proved that every participant
in $\mathcal{Q}$ correctly shared his secret share.
We note that that $s_{j\to j}$ is included in the sum for
$\gsk_{j}$ even though the encrypted form was not publicly shared.
Naturally, there is the corresponding \emph{group public key}:

\begin{equation}
    \gpk_{j} = h_{2}^{\gsk_{j}}.
\end{equation}

\noindent
Here, $\gpk_{j}$ is $P_{j}$'s portion of the master public key
and will be broadcast to all users.
Cryptographic proof that $\gpk_{j}$ is valid will be discussed
in the next section.

The threshold property along with the previous definitions give
us the following result:

\begin{equation}
    \text{msk} = \sum_{P_{j}\in\mathcal{R}} \gsk_{j}R_{j}.
\end{equation}

\noindent
These $R_{j}$ factors only depend on $\mathcal{R}$.
It follows that

\begin{equation}
    \text{mpk} = \prod_{P_{j}\in\mathcal{R}} \gpk_{j}^{R_{j}}.
\end{equation}

\noindent
This will allow partial signatures to be combined into a valid
group signature.

We now assume that $\mathcal{Q}$ wants to sign message $M$.
We let $H_{2C}(M)\in\G_{1}$ be the result of a hash-to-curve algorithm;
these will be discussed in Sec.~\ref{ssec:hash-to-curve}.
In this case, participant $P_{j}\in\mathcal{R}$ computes the
partial signature

\begin{equation}
    \sigma_{j} = \brackets{H_{2C}(M)}^{\gsk_{j}}.
\end{equation}

\noindent
For security, we should confirm

\begin{equation}
    \textsc{PairingCheck}(\sigma_{j},\bar{h}_{2},H_{2C}(M),\gpk_{j}) = 1
\end{equation}

\noindent
to ensure we have a valid signature.
If $\gpk_{j}$ is stored and we can call the hash-to-curve function $H_{2C}$,
then the only inputs will be the message $M$
and signature $\sigma_{j}$.

It is easy to compute the group signature from the partial signatures:

\begin{equation}
    \sigma = \prod_{P_{j}\in\mathcal{R}} \sigma_{j}^{R_{j}}.
\end{equation}

\noindent
The $R_{j}$ constants are defined in Eq.~\eqref{eq:Rj_coefs};
as mentioned above, the $R_{j}$ depend upon $\mathcal{R}$
and can be computed by anyone.
Because $\sigma_{j}\in\G_{1}$, signature aggregation could be carried
out by the EVM if desired; the gas cost would come from calls to
\textsc{ECAdd}, \textsc{ECMul}, and modular exponentiation,
but this would be expensive.
We now prove this is the signature corresponding to the master public key:

\begin{align}
    e(\sigma, h_{2})
        &= \prod_{P_{j}\in\mathcal{R}}
            e\parens{\sigma_{j}^{R_{j}},h_{2}}
            \nonumber\\
        &= \prod_{P_{j}\in\mathcal{R}}
            e\parens{\brackets{H_{2C}(M)}^{\text{gsk}_{j}},h_{2}^{R_{j}}}
            \nonumber\\
        &= \prod_{P_{j}\in\mathcal{R}} e\parens{H_{2C}(M),
                \brackets{h_{2}^{\text{gsk}_{j}}}^{R_{j}}} \nonumber\\
        &= e(H_{2C}(M),\text{mpk}).
\end{align}

\noindent
It follows that

\begin{equation}
    \textsc{PairingCheck}(\sigma,\bar{h}_{2},H_{2C}(M),\text{mpk}) = 1.
\end{equation}

\noindent
This allows every member of $\mathcal{R}$ to compute the signature
for the entire group while not requiring anyone to share his
signing key.
This is of utmost importance for security.



\subsubsection{Malicious Group Public Key Shares}

We now look at how to ensure the broadcast value of
$\gpk_{j}$ is valid.

Along with the $P_{j}$'s group public key $\gpk_{j}\in\G_{2}$,
there is the corresponding version in $\G_{1}$:

\begin{equation}
    \gpk_{j}^{*} = g_{1}^{\gsk_{j}}.
\end{equation}

\noindent
Note the base is $g_{1}$ and not $h_{1}$ as in the case of $\text{mpk}^{*}$
in Eq.~\eqref{eq:mpk_dual}.
Participant $P_{j}$ will publish a signature $\sigma_{j}$ of a predetermined
message $M$ for security purposes.
An initial check will ensure

\begin{equation}
    \textsc{PairingCheck}(\sigma_{j},\bar{h}_{2},H_{2C}(M),\gpk_{j}) = 1;
\end{equation}

\noindent
any validator who provides an invalid signature
is clearly malicious.

Because $P_{j}$ correctly shared his secret, $\gpk_{j}^{*}$
is public knowledge, as

\begin{equation}
    \gpk_{j}^{*} = \prod_{P_{i}\in\mathcal{Q}}F_{i}(j).
    \label{eq:gpkj_star_def}
\end{equation}

\noindent
Thus, it could be reconstructed inside a smart contract.
From there, we will ensure

\begin{equation}
    \textsc{PairingCheck}(\gpk_{j}^{*},\bar{h}_{2},g_{1},\gpk_{j}) = 1.
\end{equation}

\noindent
Requiring submission of $\gpk_{j}^{*}$ and the associated
proof could be required at the time of submission.
This would be costly, though, and we will instead allow
the other participants to prove malicious behavior.
We would prefer the entire process to be as inexpensive
as possible if there are no malicious actors.

It will be expensive to carry out this proof of malicious action.
The reason is that in order to compute $\gpk_{j}^{*}$,
the message containing the shared secrets and public coefficients
from every participant will need to be entered into a smart contract
in order to obtain the right hand side of
Eq.~\eqref{eq:gpkj_star_def}.
Because of this, there is another, potentially cheaper way to prove malicious
behavior provided a Byzantine-fault tolerant group
correctly submits $\gpk_{j}$ with valid signatures $\sigma_{j}$
and we discuss this now.

First, a participant can check to see which participants
correctly submitted $\gpk_{j}$;
this can be done using the previous method.
Let $\mathcal{R}\subseteq\mathcal{Q}$ form a Byzantine fault tolerant
majority consisting of actors who correctly shared $\gpk_{j}$;
this would imply that $\mathcal{R}$ can form a valid group signature.
Fix an honest participant $P_{v}\in\mathcal{R}$.
If $P_{i}\in\mathcal{Q}\setminus\mathcal{R}$ is malicious,
then $\braces{P_{i}}\cup\parens{\mathcal{R}\setminus \braces{P_{v}}}$
will not be able to form a valid group signature.
In this case, two lists of participants would be entered into a smart
contract: a list of valid participants followed by a list of
invalid participants.
It is straightforward to show the list of valid
participants form a valid group signature.
From there, malicious participant's signatures are included 
one at a time to show that they form invalid signatures,
giving cryptographic proof of malicious behavior.
Naturally, it would be a malicious action if false participant
lists were submitted.

We now look at the costs of both methods.
The major costs will be calling the precompiled contracts
\textsc{ECAdd}, \textsc{ECMul}, \textsc{PairingCheck},
and \textsc{ModExp} (modular exponentiation);
see Table~\ref{tab:evm_gas_cost} for the specific gas costs.
Note that the cost for \textsc{PairingCheck} comes
from our assumption that we are testing 2 pairings,
while the cost for \textsc{ModExp} comes from the fact
that all of our arguments are 256-bit (32-byte) unsigned integers.
These costs come from EIP-198\footnote{
    \url{https://github.com/ethereum/EIPs/blob/master/EIPS/eip-198.md}}
and EIP-1108\footnote{
    \url{https://github.com/ethereum/EIPs/blob/master/EIPS/eip-1108.md}}.

\input{tables/evm_cost.tex}

We begin with the standard method.
First, to compute $\gpk_{j}^{*}$, we must compute $F_{i}(j)$:

\begin{align}
    F_{i}(j) &= C_{i0}C_{i1}^{j}C_{i2}^{j^{2}}\cdots C_{it}^{j^{t}} \nonumber\\
        &= \prod_{k=0}^{t} C_{ik}^{j^{k}}.
    \label{eq:Fij_def}
\end{align}

\noindent
The cost of computing $F_{i}(j)$ is dominated by $t$ calls to \textsc{ECMul}.
Let us look at Eq.~\eqref{eq:gpkj_star_def} along with Eq.~\eqref{eq:Fij_def}:

\begin{align}
    \gpk_{j}^{*} &= \prod_{P_{i}\in\mathcal{Q}}F_{i}(j) \nonumber\\
        &= \prod_{P_{i}\in\mathcal{Q}}
                \brackets{\prod_{k=0}^{t} C_{ik}^{j^{k}}}
            \nonumber\\
        &= \prod_{k=0}^{t}
            \brackets{\prod_{P_{i}\in\mathcal{Q}} C_{ik}^{j^{k}}} \nonumber\\
        &= \prod_{k=0}^{t}
            \brackets{\prod_{P_{i}\in\mathcal{Q}} C_{ik}}^{j^{k}}
\end{align}

\noindent
Inside the brackets, we perform $n-1$ \textsc{ECAdd} operations,
and these are performed $t+1$ times.
Then, there are $t$ \textsc{ECMul} operations.
Finally, there are $t$ additional \textsc{ECAdd} operations.
Combining these, there are $t$ calls to \textsc{ECMul},
$nt + n$ calls to \textsc{ECAdd}, and one call to \textsc{PairingCheck}.
Thus, we see

\begin{equation}
    \text{Cost of Standard Proof} \sim 113000 + 4150n + 100n^{2}.
    \label{eq:cost_standard}
\end{equation}

\noindent
For $n=20$, this corresponds to $236$K gas;
the gas limit is 10M.

We now look at the cost of the group method
and start by determining the cost of the group signature.
The computation consists of forming

\begin{equation}
    \sigma = \prod_{P_{j}\in\mathcal{R}^{*}_{k}} \sigma_{j}^{R_{j}},
\end{equation}

\noindent
where $\abs{\mathcal{R}^{*}_{k}} = t+1$.
From there, we would call

\begin{equation}
    \textsc{PairingCheck}(\sigma, \bar{h}_{2}, H_{2C}(M), \text{mpk})
\end{equation}

\noindent
to ensure that this is a valid or invalid signature as appropriate.

Each group signature $\sigma$ will require forming $R_{j}$
and computing $\sigma_{j}^{R_{j}}$.
After computing $R_{j}$, we require one \textsc{ECMul} call
to compute $\sigma_{j}^{R_{j}}$.
The expensive part turns out to be forming $R_{j}$,
and we focus on its straightforward computation.
We recall from Eq.~\eqref{eq:Rj_coefs} that

\begin{equation}
    R_{j} = \prod_{\substack{P_{k}\in\mathcal{R} \\ k\ne j}} \frac{k}{k-j}.
\end{equation}

\noindent
Naively, we would need to compute $t$ finite field inversions,
which corresponds to $t$ \textsc{ModExp} calls, to compute
$R_{j}$.
This must be performed $t+1$ times to form $\sigma$
for a total cost of $O(n^{2})$ \textsc{ModExp} calls.
Looking at the Table~\ref{tab:evm_gas_cost}, we notice this gas cost
is too expensive.
If we precompute these inverses and include them
in the function call, then the overall gas cost will be greatly reduced.
At the beginning of the call, we check to make sure that the submitted
inverses are valid; if they are valid, we proceed with the
accusation, and if they are invalid, we stop.
See Alg.~\ref{alg:grpsig_malicious} for the complete description.
There are $t+1$ \textsc{ECMul} operations, $t$ \textsc{ECAdd} operations,
and one pairing check per signature check.
Thus, the cost to check one signature is

\begin{equation}
    \text{Cost of Group Proof} \sim 113000 + 4100n.
    \label{eq:cost_group}
\end{equation}

\noindent
For $n=20$, this corresponds to $195$K gas.
This is slightly more efficient than computing $\gpk_{j}^{*}$,
but this is not the whole story.

We now take time to make a valid comparison between the
standard accusation (computing $\gpk_{j}$ and group accusation
(computing group signatures).
First, the standard accusation additional gas cost associated
with submitted and verifying public information related to the
encrypted shares and public coefficients;
we do not include an analysis of these costs at this time
and will look at doing that in the future.

By directly comparing Eq.\eqref{eq:cost_standard} and
Eq.~\eqref{eq:cost_group}, it would appear that the group accusation
is more efficient.
Of course, we must remember if there are $m$ malicious participants,
then there are $m$ computations of $\gpk_{j}$,
while there are $m+1$ computations of group signatures.
Furthermore, group signatures have the additional restriction that
there \emph{must} be a honest majority.
Thus, for a valid comparison to be made, we must know the
number of participants as well as the total number of malicious actors.
Tab.~\ref{tab:gpkj_gas_cost} lists the gas cost for select values.
Note that the table does not include the cost associated
with entering the necessary information for the standard accusation;
even so, this is a reasonable estimate.

\begin{table}
\centering
\begin{tabular}{|c||r|r||r|r||r|r|}
\hline
  & \multicolumn{2}{|c||}{$n = 10$} & \multicolumn{2}{|c||}{$n = 20$} &
    \multicolumn{2}{|c|}{$n = 30$} \\
\hline
    $m$ & \multicolumn{1}{|c|}{Std} & \multicolumn{1}{|c||}{Grp} &
          \multicolumn{1}{|c|}{Std} & \multicolumn{1}{|c||}{Grp} &
          \multicolumn{1}{|c|}{Std} & \multicolumn{1}{|c||}{Grp} \\
\hline
    1 &  164 &  308 &  236 &  390 &  328 &  472 \\
    2 &  329 &  462 &  472 &  585 &  655 &  708 \\
    3 &  494 &  616 &  708 &  780 &  982 &  944 \\
    4 &  658 &      &  944 &  975 & 1310 & 1180 \\
    5 &  822 &      & 1180 & 1170 & 1638 & 1416 \\
    6 &  987 &      & 1416 & 1365 & 1965 & 1652 \\
    7 & 1152 &      & 1652 &      & 2292 & 1888 \\
\hline
\end{tabular}
\caption[Malicious $\gpk_{j}$ Accusation Gas Cost]{
This is the estimated gas cost comparing the standard and group
$\gpk_{j}$ accusation. The cost is listed in thousands of gas
for the Ethereum Virtual Machine.
Here, $n$ is the number of total participants and $m$ is the number
of malicious actors.
There are no entries in the group method when group accusations
are not possible due to no BFT majority.
}
\label{tab:gpkj_gas_cost}
\end{table}


In all cases, multiple malicious actors could be accused
in a single transaction and still stay well below the 10M gas limit.
Due to the different cost functions, the exact method will depend
on how many validators are dishonest and the total number of participants.
The standard accusation always works even though the cost function
grows more quickly.

\input{algs/gpkj_algs.tex}


\subsection{Hash-to-curve Functions}
\label{ssec:hash-to-curve}

The contents of this section should not be considered a full proof
of security but are for implementation reference.
Formal treatment of the security proofs for hash-to-curve
functions are handled in referenced works.

We turn our attention to the hash-to-curve functions
implemented in our system.
The contents of this section are described hereafter.
First, we review a mechanism often employed to solve this problem.
Then, we briefly discuss why this solution is not ideal in our setting.
Finally, we introduce the basis of our implementation and proceed into the
definition of the algorithms used in our system after a brief review of
some required mathematical operations.
All algorithms may be found at the end of this section.

The seemingly standard, non-deterministic method to perform a hash-to-curve
operation in the elliptic curve setting is to use the ``MapToGroup''
method as presented in the original BLS short signature
paper~\cite{boneh2001short}.
In this method values are hashed into
$\F_{p}$ by modular arithmetic with a concatenated counter.
This counter starts at a fixed value, and is incremented until
a value is found that creates a valid point on the specified
elliptic curve.
While this may be sufficient in some instances, we prefer 
deterministic methods due to the need for bounded
computational overhead.
In our system, the need for bounded computational
overhead arises from a desire to allow an Ethereum smart contract to
perform the hash-to-curve operations with bounded gas consumption.
Deterministic methods also have the benefit of minimizing
side-channel attacks in those
algorithms that require such protection.

The hash-to-curve implementation selected allows the problem space to be
divided into independent problems such that their solutions may be composed.
Specifically, we first hash to the base field
(hash map $\mathfrak{h}:\braces{0,1}^{*}\to \F$)
and then find a deterministic map from the base field
to the elliptic curve (function $f:\F\to E(\F)$).
This approach allows for a separation of concerns and has become a
standard approach to the problem of hashing into an elliptic curve
~\cite{icart2009hash,ft2012bnhashtocurve,boneh2019h2cBLS12}.
Although this strategy does offer a simplified view of the problem,
mapping from the base field to the elliptic curve is nontrivial.
Additionally, it is frequently the case that $f$ is not surjective, but
we can overcome this limitation to obtain a surjective hash-to-curve algorithm
under easily-satisfied
conditions~\cite{tibouchi2014elligator,ft2012bnhashtocurve}.
Specifically, we use domain-separation in order to obtain independent hash
functions.
These independent hash functions allow us overcome a non-surjective
mapping.
We will fully address this concern later in this document.

Our discussion of hashing functions follow
\cite{ft2012bnhashtocurve,boneh2019h2cBLS12}.
We ask the reader to note that in the remaining work of this section,
we view the group $E(\F_{p})$ additively unless otherwise specified.
We highlight this fact because this is different from the multiplicative
notation used previously in this document.
Before we define our implementation of the hash-to-curve algorithms employed,
we feel a review of the mathematics would benefit the reader.
Thus, we first
review the mathematics necessary to perform the hash-to-curve operations.
After this review, we present the algorithm for hashing to $\G_{1}$
and then present the algorithm for hashing to $\G_{2}$.

For reference, the full \textsc{HashToG1} algorithm can be
found in Alg.~\ref{alg:hash-to-G1} and the full algorithm for \textsc{HashToG2}
can be found in Alg.~\ref{alg:hash-to-G2}.



\subsubsection{Inverses, Square Roots, and Legendre Symbols
    in $\F_{p}$}
\label{sssec:finite_math_Fp}

In this section we review some finite field mathematics that are
important in our hash-to-curve setting.

We begin by reviewing inversion in $\F_{p}$.
First, we recall $\F_{p}^{*} \equiv \F_{p}\setminus\braces{0}$,
the nonzero elements of our finite field.
For $a\in\F_{p}^{*}$, we have Euler's formula:

\begin{equation}
    a^{p-1} = 1 \mod p.
\end{equation}

\noindent
This implies $a^{-1} \mod p = a^{p-2}$.
We acknowledge more efficient methods for computing modular inverses are
possible, but exponentiation can easily be performed in constant time, which
is a goal of our implementation. This concludes our discussion of computing
inverses in $\F_{p}$.

We now review the mechanisms for
computing Legendre symbols.
We recall that $a\in\F_{p}^{*}$ is a \emph{quadratic residue}
if there is $x\in\F_{p}$ such that $x^{2} = a \mod p$;
otherwise, $a$ is a called a \emph{quadratic nonresidue}.
This allows us to define the Legendre symbol~\cite{imc2008}:

\begin{equation}
    \chi_{p}(a) \equiv \begin{cases}
        1 \quad& \text{if there $a$ is a quadratic residue modulo $p$.} \\
        -1 \quad& \text{if there $a$ is a quadratic nonresidue modulo $p$.} \\
        0 \quad& \text{if $p \mid a$}
    \end{cases}
\end{equation}

\noindent
The Legendre symbol has a simple formula:

\begin{equation}
    \chi_{p}(a) = a^{\frac{p-1}{2}}.
\end{equation}

\noindent
This formula holds even when $a=0$.
This concludes our discussion of computing the Legendre Symbol in $\F_{p}$.

Computing square roots is more challenging.
We now review the mechanism of computing square roots.
In our case $p = 3\mod 4$, so there is a simple
formula to solve $x^{2} = a$:

\begin{equation}
    x = a^{\frac{p+1}{4}}.
\end{equation}

\noindent
This relies on the assumption that $a$ is a quadratic residue,
so $a^{\frac{p-1}{2}} = 1$.
This concludes our discussion of computing the square roots in $\F_{p}$.

Taking inverses, Legendre symbols, and square roots
gives us the necessary tools to hash to $\G_{1}$.
This ends the preliminary work necessary to develop
\textsc{HashToG1}.


\subsubsection{Inverses, Square Roots, and Legendre Symbols
    in $\F_{p^{2}}$}
\label{sssec:finite_math_Fp2}

The previous section reviewed the mathematics of inverses, square roots, and
Legendre Symbols in $\F_{p}$.
We must be able to perform the same operations
in $\F_{p^{2}}$ if we wish to hash to $\G_{2}$.
While computing square roots and
inverses may be familiar to the reader in $\F_{p}$, the operations require
special handling in $\F_{p^{2}}$.
We review these mechanisms now.

We first address the problem of computing inverses in $\F_{p^{2}}$.
Our discussion and methods follow~\cite{adj2012sqrtEvenExt},
and we present their general results applied to our particular case.
As mentioned above, the setting of our elliptic curve is $p = 3\mod 4$.
This implies $-1$ is a quadratic nonresidue.
Thus, we have the following isomorphism:

\begin{equation}
    \F_{p^{2}} \simeq \F_{p}[i]/\parens{i^{2}+1}.
\end{equation}

\noindent
This is the analogous to constructing the complex numbers $\C$ from
the real numbers $\R$.
Our implementation uses this construction. We may use
this construction to compute an efficient inversion as follows:

\begin{equation}
    \parens{a_{0} + a_{1}i}^{-1} = \frac{a_{0}-a_{1}i}{a_{0}^{2}+a_{1}^{2}}.
\end{equation}

\noindent
The main computational cost of this operation is the inversion of an element
in $\F_{p}$.
This concludes our discussion of computing inverses in $\F_{p^{2}}$.

We now begin our discussion of computing the Legendre Symbol in $\F_{p^{2}}$
If $a = a_{0} + a_{1}i\in\F_{p^{2}}$, then $a$ is a
quadratic residue in $\F_{p^{2}}$ if and only if $a_{0}^{2} + a_{1}^{2}$
is a quadratic residue in $\F_{p}$.
The main computational cost of this operation is the computation of the Legendre
symbol of an element in $F_{p}$.
From the above we have the Legendre symbol in $\F_{p^{2}}$,
which we denote as \textsc{LegendreFP2} or $\chi_{p^{2}}(\cdot)$;
this particular algorithm is presented in
Alg.~\ref{alg:legendre-fp2}.
This concludes our discussion of computing the Legendre Symbol in
$\F_{p^{2}}$.

We will now look at computing square roots in $\F_{p^{2}}$.
The main idea is to find $b\in\F_{p^{2}}$ and odd $s$ such
$b^{2}a^{s} = 1$.
In this case, we see

\begin{align}
    \brackets{ba^{\frac{s+1}{2}}}^{2} &= b^{2}a^{s+1} \nonumber\\
        &= a.
\end{align}

\noindent
If we set

\begin{align}
    b &= \parens{1 + a^{\frac{p-1}{2}}}^{\frac{p-1}{2}} \nonumber\\
    s &= \frac{p-1}{2},
\end{align}

\noindent
then when $b\ne0$, we have $b^{2}a^{s} = 1$.
When $b=0$, we have $a^{\frac{p-1}{2}}=-1$.
In this case, our square root is $ia^{\frac{p+1}{4}}$.
This procedure is formally presented in~Alg.~\ref{alg:sqrt-fp2}.
The main computational cost of this operation is two exponentiations
in $\F_{p^{2}}$.
This concludes our discussion of computing square roots in $\F_{p^{2}}$
and the preliminary review of those operations
necessary to develop \textsc{HashToG2}.


\subsubsection{Hashing to Base}
\label{sssec:hash-to-base}

In this section we will describe the hash-to-base operation.
We begin by discussing the construction of a random oracle into
$\F_{p}$ using a single cryptographic hash function with
domain separation.
We then bound the deviation from uniformity in this operation.

In the following $H$ is a $256$-bit hash function.
In our implementation we use \textsc{Keccak256}.
Note this is the \textsc{Sha3} variant used by Ethereum
that differs from the NIST approved \textsc{Sha3} hash function due to a
difference in the handling of padding.

Let $H$ act as a random oracle.
We map from $H$ to $\Z$ by interpreting the output of a
cryptographic hash function as a big endian unsigned integer.
This can be seen in lines 4 and 5 of $\textsc{HashToBase}$
as the function named $\texttt{b2u}$.
Although we may naively map from $\Z$ to $\Z_{p}$ by simply taking the
output of $H$ modulo $p$, this would be insecure in our setting due to the
nonuniformity of this operation.
In order to ensure the mapping from $\Z$ to $\Z_{p}$ is more
uniformly distributed, domain separation is utilized.
Specifically, we use domain separation in order obtain independent hash
functions from $H$ through concatenation of constants with the message being
hashed.
These independent hash functions allow us to create a secure 512-bit random
number from a single cryptographic hash function.
The full explanation of this operation is below.

Let $\textsc{HashToBase}:\braces{0,1}^{*}\times\braces{0,1}^{8}\times
\braces{0,1}^{8}\to\F_{p}$ denote our random oracle into the
underlying field $\F_{p}$.
In the $\textsc{HashToBase}$ operation the first component corresponds to the
underlying message being hashed, while the last two elements provide the
necessary domain separation.
See Alg.~\ref{alg:hash-to-fp} for the full implementation.

Due to the fact $p$ is prime and not a power of 2, there will be some
nonuniformity in the resulting distribution of \textsc{HashToBase}.
We investigate this nonuniformity now.

First, we assume that $H:\braces{0,1}^{*}\to\Z_{N}$
is a random oracle and $p\in\braces{1,2,\cdots,N-1}$.
We want to determine how much $H(m)\mod p$ deviates from uniformity.
We restrict ourselves to the case when $p\nmid N$.
Let

\begin{equation}
    N = qp + r
\end{equation}

\noindent
with $0\le r< p$ and $q\ge1$.
Because $p\nmid N$, we have $r\ge1$.
Let $X$ be uniformly distributed on $\Z_{N}$ and set $X_{p} = X \mod p$.
Furthermore, we let $U_{p}$ be the uniform distribution modulo $p$.
Then

\begin{equation}
    \mathcal{P}\parens{X_{p}\in\braces{0,\cdots,r-1}} = \frac{q+1}{N}
\end{equation}

\noindent
and

\begin{equation}
    \mathcal{P}\parens{X_{p}\in\braces{r,\cdots,p-1}} = \frac{q}{N}.
\end{equation}

\noindent
Here, $\mathcal{P}$ denotes the probability of an event occurring.
We now determine the deviation of $X_{p}$ from the uniform
distribution $U_{p}$:

\begin{align}
    \Delta\parens{X_{p},U_{p}} &\equiv
        \sum_{k=0}^{p-1} \abs{\mathcal{P}(X_{p}=k) - \mathcal{P}(U_{p}=k)}
        \nonumber\\
    &= \sum_{u=0}^{r-1} \abs{\frac{q+1}{N} - \frac{1}{p}} +
        \sum_{u=r}^{p-1} \abs{\frac{q}{N} - \frac{1}{p}} \nonumber\\
    &= r\frac{qp + p - N}{pN} + \parens{p-r}\frac{N-qp}{pN}
        \nonumber\\
    &\le \frac{p}{N}.
\end{align}

\noindent
From this, we see that if $p$ is a $k$-bit prime and
$N = 2^{k+\ell}$, then the deviation from uniformity is $\le2^{-\ell}$.

In our case, $p$ is a 254-bit prime and we concatenate the output
of independent hash functions in order to have a uniformly distributed
512-bit output.
From the above it may be seen that \textsc{HashToBase} produces outputs whose
deviation from uniformity is less than $2^{-258}$.
This deviation is sufficiently small as to not be of concern.
Further work is required to more formally prove this assumption, but
this work is not included at this time.


\subsubsection{Base to $\G_{1}$}
\label{sssec:base-to-G1}

In this section we discuss the construction of a hash-to-G1 algorithm.
We begin by noting the non-surjective nature for many functions
$f:\F_{p}\to E(\F_{p})$.
Then, we cite a known solution to this problem and provide necessary
mathematics to understand the fundamental operation that
overcomes the problem.
Next, we discuss the actual implementation.
Lastly, we discuss specific exclusion of three points from the allowable
outputs of this algorithm for security purposes.

Let us now suppose that we have a hash function
$\mathfrak{h}:\braces{0,1}^{*}\to\F_{p}$ and a
deterministic map $f:\F_{p}\to E(\F_{p})$.
As has been previously stated, in many
situations~\cite{icart2009hash,ft2012bnhashtocurve,boneh2019h2cBLS12}
$f$ is not surjective.
That is, there are points on the elliptic curve $E(\F_{p})$
(sometimes a nontrivial fraction) which cannot be reached by $f$.
Even so, if we use domain separation to construct independent hash functions,
$\mathfrak{h}_{1}, \mathfrak{h}_{2}:\braces{0,1}^{*}\to\F$,
then

\begin{equation}
    F(\texttt{m}) = f(\mathfrak{h}_{1}(\texttt{m}))
                    + f(\mathfrak{h}_{2}(\texttt{m}))
\end{equation}

\noindent
is indistinguishable from a random oracle on $E(\F_{p})$
under certain restrictions on $f$.
See~\cite{ft2012bnhashtocurve,tibouchi2014elligator} for details.

We now turn our attention to determining $f:\F_{p}\to E(\F_{p})$.
Finding a map $f:\F_{p}\to E(\F_{p})$ is involved.
Our BN curve has the form

\begin{align}
    E:y^{2} &= g(x) \nonumber\\
            &= x^{3} + b.
\end{align}

\noindent
From~\cite{ft2012bnhashtocurve}, it possible to show there
are $x_{1},x_{2},x_{3},y\in\F_{p}$ such that

\begin{equation}
    g(x_{1})g(x_{2})g(x_{3}) = y^{2}.
\end{equation}

\noindent
When $y\ne0$, quadratic reciprocity implies that $g(x_{i})$ is square
for some $i$; that is, for some $i$ we have
$(x_{i},\sqrt{g(x_{i})})\in E(\F_{p})$.
For uniqueness, we choose the smallest $i$.
We will use the construction of~\cite{ft2012bnhashtocurve}
to determine such points with some modifications based
on work in~\cite{boneh2019h2cBLS12}.
The exact algorithm can be found in Alg.~\ref{alg:base-to-G1}.

One of the main difficulties is determining $i$ in such a way
as to not leak information; because of this, we do not wish to
rely upon if statements.
In~\cite{ft2012bnhashtocurve}, they suggested the function

\begin{equation}
    \psi(r_{1},r_{2}) = \brackets{\parens{r_{1}-1}r_{2}\mod3} + 1.
    \label{eq:ftAB}
\end{equation}

\noindent
This function works under the assumption that modular
arithmetic always returns positive integers.
This is not always the case in programming languages;
in particular, it does not hold in \textsc{Go} (Golang), the language
we use to implement these algorithms.
To circumvent this, we use the following function,
which is implemented in Line~\ref{alg_line:new_coef_func}
of Alg.~\ref{alg:base-to-G1}:

\begin{equation}
    \parens{r_{1}-1}\parens{r_{2}-3}/4 + 1.
    \label{eq:newAB}
\end{equation}

In~\cite{ft2012bnhashtocurve}, the authors encountered an issue when $t=0$.
Their solution was to define

\begin{equation}
    \textsc{BaseToG1}(0) = \parens{\frac{-1+\sqrt{-3}}{2}, \sqrt{1+b}}.
    \label{eq:baseToG1zero}
\end{equation}

\noindent
Wahby and Boneh~\cite{boneh2019h2cBLS12} suggest another
method, which we implement, in order to have a more efficient
algorithm.
This implementation also affords the benefit of not needing to
handle the case of $t=0$ separately.
This can be seen in Line~\ref{alg_line:alpha_def}
of Alg.~\ref{alg:base-to-G1} where we define $\alpha$ as the inverse
of a value which depends on $t\in\F_{p}$.
When $t=0$, we compute $\alpha = 0$
because we compute inversions via exponentiation.
Thus, the computational convention $0^{-1} = 0 \mod p$ is established.
This leads to the same result as in Eq.~\eqref{eq:baseToG1zero}
without special handling.

We have cryptographic insecurity when
$\textsc{HashToG1}(\texttt{m}) = g_{1}^{\alpha}$ for known $\alpha$.
Note we briefly switch back to multiplicative notation at this time.
In practice, it is likely this insecurity will only be known when
$\alpha\in\braces{-1,0,1}$ or is sufficiently close to these values.
We assume we may not fix simple proximity to these values, and thus only
explicitly address the case of $\alpha\in\braces{-1,0,1}$.
For clarity, this notation specifies the hash function outputs of
the identity element, the generator, or the generator's
inverse.
Due to the concerns around the use of these points, we will not allow these
individual cases.
In order to enforce this requirement we include a
\textsc{SafePointCheck} in \textsc{HashToG1};
see Line~\ref{alg_line:safe_point_check} in Alg.~\ref{alg:hash-to-G1}.
This function checks the point returned from the hash function for
equivalence with the identity element OR equivalence of the point's
$x$ coordinate with $1$.
If either of these conditions is true, an error is raised.
Although the probability of a hash mapping to these points is small, this
error must be handled appropriately to prevent an attacker from causing
unexpected errors in a remote system due to specially crafted messages.


\subsubsection{Base to $\G_{2}$}
\label{sssec:base-to-G2}

We now take up the slightly more challenging possibility
of computing a hash function to $\G_{2}$.
Fortunately, we may utilize many of
the previously described operations.
Thus, we do not repeat those explanations and only discuss
those operations that differ.
We would like to remind the reader that the results in
Sec.~\ref{sssec:finite_math_Fp2} allow
us to compute Legendre symbols and square roots in $\F_{p^{2}}$.

Using the derivation of~\cite[Section 3]{boneh2019h2cBLS12},
we set $u_{0}=1$ as in $\G_{1}$ to obtain

\begin{align}
    x_{1} &= \frac{\sqrt{-3}-1}{2} - \frac{t^{2}\sqrt{-3}}{t^{2}+g'(1)}
        \nonumber\\
    x_{2} &= -1 - x_{2} \nonumber\\
    x_{3} &= 1 - \frac{\parens{t^{2}+g'(1)}^{2}}{3t^{2}},
\end{align}

\noindent
for $t\in\F_{p^{2}}$.
Here, we have

\begin{equation}
    E':y^{2} = g'(x) = x^{3} + b',
\end{equation}

\noindent
where $b' = b/\xi = 3/\parens{i+9}$.
With this choice, $-g'(1)$ is a residue in $\F_{p^{2}}$.
This ensures that $t\in\{0,\pm\sqrt{-g'(1)}\}$
implies $x_{1}$ is a valid point on the curve;
thus, all inputs result in valid outputs.
See Alg.~\ref{alg:base-to-twist} for the algorithm.
Thus, our hash functions to $\G_{1}$ and $\G_{2}$ are
essentially the same.

At this point, we have successfully mapped into $E'$;
however, our goal is to map into $\G_{2}$.
From~\cite{bnCurves}, we know $\abs{E'(\F_{p^{2}})} = q\parens{2p-q}$.
This gives a cofactor $r = 2p-q$ because $p\nmid q$.
We take care of this by clearing the cofactor.
See Alg.~\ref{alg:hash-to-G2} for the full hash algorithm.

\clearpage

\input{algs/hash_to_G1.tex}

\input{algs/gfp2_algs.tex}

\input{algs/hash_to_G2.tex}

