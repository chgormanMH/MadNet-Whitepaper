\subsection{Group Signatures}
\label{ssec:crypto_group_sig}

\subsubsection{Constructing the Master Public Key}

The goal of our consensus algorithm is to enable a Byzantine-fault
tolerant subgroup to cryptographically sign on behalf of the entire
group without requiring every individual group member to sign.
This is enabled by signature aggregation in the appropriate way.

We let $\mathcal{Q}$ be the collection of qualified
actors who correctly shared their secrets and
$\mathcal{R}\subseteq\mathcal{Q}$ such that
$\abs{\mathcal{R}} = t+1$; thus, $\mathcal{R}$ is a
Byzantine-fault tolerant subgroup.
As discussed in~\cite{gennaro3revisiting,gennaro1999secure,ethdkg},
in order to ensure that no bad actors gain any information
about the master public key and not be able to change
its underlying probability distribution, we require
$h_{1}\in\G_{1}$ such that $\dlog_{g_{1}}h_{1}$ is unknown.
We also let $h_{2}\in\G_{2}$ be a generator.

The individual shared secrets $s_{i}$ allow us to define the
\emph{master secret key} $\text{msk}$:

\begin{equation}
    \text{msk} = \sum_{P_{i}\in\mathcal{Q}} s_{i}.
\end{equation}

\noindent
This gives us the \emph{master public key} $\text{mpk}$:

\begin{align}
    \text{mpk} &= h_{2}^{\text{msk}} \nonumber\\
        &= \prod_{P_{i}\in\mathcal{Q}} h_{2}^{s_{i}}.
\end{align}

\noindent
Because everyone in $\mathcal{Q}$ correctly shared his secret,
a Byzantine-fault tolerant subgroup $\mathcal{R}$ can
correctly obtain the secret $s_{i}$ via Lagrange interpolation:

\begin{align}
    s_{i} &= \sum_{P_{j}\in\mathcal{R}} s_{i\to j} R_{j} \nonumber\\
    R_{j} &= \prod_{\substack{P_{k}\in\mathcal{R} \\ k\ne j}} \frac{k}{k-j}.
    \label{eq:Rj_coefs}
\end{align}

\noindent
This would allow us to recover the secret $s_{i}$ should $P_{i}$
fail to share $h_{1}^{s_{i}}$ below; however,
we take a stricter response and would view failure to share
as malicious activity leading to stake slashing.

We now proceed to compute $\text{mpk}$.
Let

\begin{equation}
    \pi(h_{1}^{s_{i}}) = \textsc{DLEQ}(
        g_{1},g_{1}^{s_{i}},h_{1},h_{1}^{s_{i}},s_{i})
\end{equation}

\noindent
be the zk-proof that $h_{1}^{s_{i}}$ is $P_{i}$'s portion
of the master public key (technically part of $\text{mpk}^{*}$
as defined below).
Because $C_{i0} = g_{1}^{s_{i}}$ is public knowledge
and $P_{i}$ correctly shared his secret $s_{i}$, it is possible
to publicly verify $h_{1}^{s_{i}}$.
Additionally, $P_{i}$ will publish $h_{2}^{s_{i}}$ so that
we can ensure

\begin{equation}
    \textsc{PairingCheck}(h_{1}^{s_{i}},\bar{h}_{2},h_{1},h_{2}^{s_{i}})
        = 1.
\end{equation}

\noindent
This will be called by a smart contract.
Thus, failure of $P_{i}$ to publish $h_{1}^{s_{i}}$,
a valid proof $\pi(h_{1}^{s_{i}})$, and the corresponding
$h_{2}^{s_{i}}$ amounts to misbehavior which will
lead to a fine.

The Ethereum smart contract will store $h_{1}^{s_{i}}$ from
all participants and broadcast
$h_{1}^{s_{i}}$, $\pi(h_{1}^{s_{i}})$, and $h_{2}^{s_{i}}$.
From here, any participant will be able to submit

\begin{equation}
    \text{mpk} = \prod_{P_{i}\in\mathcal{Q}} h_{2}^{s_{i}}
\end{equation}

\noindent
to the smart contract.
Because $\braces{h_{1}^{s_{i}}}_{i\in\mathcal{Q}}$ are stored
and valid, the contract can construct

\begin{equation}
    \text{mpk}^{*} = \prod_{P_{i}\in\mathcal{Q}} h_{1}^{s_{i}}
    \label{eq:mpk_dual}
\end{equation}

\noindent
and call

\begin{equation}
    \textsc{PairingCheck}(\text{mpk}^{*},\bar{h}_{2},h_{1},\text{mpk})
\end{equation}

\noindent
to ensure $\text{mpk}$ is valid.
The master public key can then be stored publicly and used
for group signature verification.



\subsubsection{Constructing Group Signatures}

At this point, we have successfully constructed the master public
key $\text{mpk}$ for $\mathcal{Q}$ and distributed
the master secret key $\text{msk}$ among the members of $\mathcal{Q}$.
We now turn our attention to constructing group
signatures from partial signatures.

Each participant $P_{j}\in\mathcal{Q}$ has a portion of the
master secret key; this is portion is called the \emph{group secret
key}:

\begin{equation}
    \gsk_{j} = \sum_{P_{i}\in\mathcal{Q}} s_{i\to j}.
\end{equation}

\noindent
This is possible because we proved that every participant
in $\mathcal{Q}$ correctly shared his secret share.
We note that that $s_{j\to j}$ is included in the sum for
$\gsk_{j}$ even though the encrypted form was not publicly shared.
Naturally, there is the corresponding \emph{group public key}:

\begin{equation}
    \gpk_{j} = h_{2}^{\gsk_{j}}.
\end{equation}

\noindent
Here, $\gpk_{j}$ is $P_{j}$'s portion of the master public key
and will be broadcast to all users.
Cryptographic proof that $\gpk_{j}$ is valid will be discussed
in the next section.

The threshold property along with the previous definitions give
us the following result:

\begin{equation}
    \text{msk} = \sum_{P_{j}\in\mathcal{R}} \gsk_{j}R_{j}.
\end{equation}

\noindent
These $R_{j}$ factors only depend on $\mathcal{R}$.
It follows that

\begin{equation}
    \text{mpk} = \prod_{P_{j}\in\mathcal{R}} \gpk_{j}^{R_{j}}.
\end{equation}

\noindent
This will allow partial signatures to be combined into a valid
group signature.

We now assume that $\mathcal{Q}$ wants to sign message $M$.
We let $H_{2C}(M)\in\G_{1}$ be the result of a hash-to-curve algorithm;
these will be discussed in Sec.~\ref{ssec:hash-to-curve}.
In this case, participant $P_{j}\in\mathcal{R}$ computes the
partial signature

\begin{equation}
    \sigma_{j} = \brackets{H_{2C}(M)}^{\gsk_{j}}.
\end{equation}

\noindent
For security, we should confirm

\begin{equation}
    \textsc{PairingCheck}(\sigma_{j},\bar{h}_{2},H_{2C}(M),\gpk_{j}) = 1
\end{equation}

\noindent
to ensure we have a valid signature.
If $\gpk_{j}$ is stored and we can call the hash-to-curve function $H_{2C}$,
then the only inputs will be the message $M$
and signature $\sigma_{j}$.

It is easy to compute the group signature from the partial signatures:

\begin{equation}
    \sigma = \prod_{P_{j}\in\mathcal{R}} \sigma_{j}^{R_{j}}.
\end{equation}

\noindent
The $R_{j}$ constants are defined in Eq.~\eqref{eq:Rj_coefs};
as mentioned above, the $R_{j}$ depend upon $\mathcal{R}$
and can be computed by anyone.
Because $\sigma_{j}\in\G_{1}$, signature aggregation could be carried
out by the EVM if desired; the gas cost would come from calls to
\textsc{ECAdd}, \textsc{ECMul}, and modular exponentiation,
but this would be expensive.
We now prove this is the signature corresponding to the master public key:

\begin{align}
    e(\sigma, h_{2})
        &= \prod_{P_{j}\in\mathcal{R}}
            e\parens{\sigma_{j}^{R_{j}},h_{2}}
            \nonumber\\
        &= \prod_{P_{j}\in\mathcal{R}}
            e\parens{\brackets{H_{2C}(M)}^{\text{gsk}_{j}},h_{2}^{R_{j}}}
            \nonumber\\
        &= \prod_{P_{j}\in\mathcal{R}} e\parens{H_{2C}(M),
                \brackets{h_{2}^{\text{gsk}_{j}}}^{R_{j}}} \nonumber\\
        &= e(H_{2C}(M),\text{mpk}).
\end{align}

\noindent
It follows that

\begin{equation}
    \textsc{PairingCheck}(\sigma,\bar{h}_{2},H_{2C}(M),\text{mpk}) = 1.
\end{equation}

\noindent
This allows every member of $\mathcal{R}$ to compute the signature
for the entire group while not requiring anyone to share his
signing key.
This is of utmost importance for security.



\subsubsection{Malicious Group Public Key Shares}

We now look at how to ensure the broadcast value of
$\gpk_{j}$ is valid.

Along with the $P_{j}$'s group public key $\gpk_{j}\in\G_{2}$,
there is the corresponding version in $\G_{1}$:

\begin{equation}
    \gpk_{j}^{*} = g_{1}^{\gsk_{j}}.
\end{equation}

\noindent
Note the base is $g_{1}$ and not $h_{1}$ as in the case of $\text{mpk}^{*}$
in Eq.~\eqref{eq:mpk_dual}.
Participant $P_{j}$ will publish a signature $\sigma_{j}$ of a predetermined
message $M$ for security purposes.
An initial check will ensure

\begin{equation}
    \textsc{PairingCheck}(\sigma_{j},\bar{h}_{2},H_{2C}(M),\gpk_{j}) = 1;
\end{equation}

\noindent
any validator who provides an invalid signature
is clearly malicious.

Because $P_{j}$ correctly shared his secret, $\gpk_{j}^{*}$
is public knowledge, as

\begin{equation}
    \gpk_{j}^{*} = \prod_{P_{i}\in\mathcal{Q}}F_{i}(j).
    \label{eq:gpkj_star_def}
\end{equation}

\noindent
Thus, it could be reconstructed inside a smart contract.
From there, we will ensure

\begin{equation}
    \textsc{PairingCheck}(\gpk_{j}^{*},\bar{h}_{2},g_{1},\gpk_{j}) = 1.
\end{equation}

\noindent
Requiring submission of $\gpk_{j}^{*}$ and the associated
proof could be required at the time of submission.
This would be costly, though, and will instead allow
the other participants to prove malicious behavior.
We would prefer the entire process to be as inexpensive
as possible should there be no malicious actors.

It will be expensive to carry out this proof of malicious action.
The reason is that in order to compute $\gpk_{j}^{*}$,
the message containing the shared secrets and public coefficients
from every participant will need to be entered into a smart contract
in order to obtain the right hand side of
Eq.~\eqref{eq:gpkj_star_def}.
Because of this, there is another, cheaper way to prove malicious
behavior provided a Byzantine-fault tolerant group
correctly submits $\gpk_{j}$ with valid signatures $\sigma_{j}$
and we discuss this now.

First, a participant can go through all possible subgroups
$\mathcal{R}\subseteq\mathcal{Q}$ searching for a subset
whose signatures can be combined to form a valid
group signature.
Once a valid subset $\mathcal{R}$ is found,
we can then choose $P_{i}\in\mathcal{Q}\setminus\mathcal{R}$
in order to determine if
$\braces{P_{i}}\cup\parens{\mathcal{R}\setminus \braces{P_{v}}}$
can form a valid signature;
if it does not, then $P_{i}$ is malicious.
Here $P_{v}\in\mathcal{R}$ is a fixed.
In this case, two lists of participants would be entered into a smart
contract: a list of valid participants followed by a list of
invalid participants.
It is straightforward to show the list of valid
participants form a valid group signature.
From there, malicious participant's signatures are included 
one at a time to show that they form invalid signatures,
giving cryptographic proof of malicious behavior.
Naturally, it would be a malicious action if false participant
lists were submitted.

We now look at the costs of both methods.
The major costs will be calling the precompiled contracts
\textsc{ECAdd}, \textsc{ECMul}, \textsc{PairingCheck},
and \textsc{ModExp} (modular exponentiation);
see Table~\ref{tab:evm_gas_cost} for the specific gas costs.
Note that the cost for \textsc{PairingCheck} comes
from our assumption that we are testing 2 pairings,
while the cost for \textsc{ModExp} comes from the fact
that all of our arguments are 256-bit (32-byte) unsigned integers.
These costs come from EIP-198\footnote{
    \url{https://github.com/ethereum/EIPs/blob/master/EIPS/eip-198.md}}
and EIP-1108\footnote{
    \url{https://github.com/ethereum/EIPs/blob/master/EIPS/eip-1108.md}}.

\input{tables/evm_cost.tex}

We begin with the standard method.
First, to compute $\gpk_{j}^{*}$, we must compute $F_{i}(j)$:

\begin{align}
    F_{i}(j) &= C_{i0}C_{i1}^{j}C_{i2}^{j^{2}}\cdots C_{it}^{j^{t}} \nonumber\\
        &= \prod_{k=0}^{t} C_{ik}^{j^{k}}.
    \label{eq:Fij_def}
\end{align}

\noindent
The cost of computing $F_{i}(j)$ is dominated by $t$ calls to \textsc{ECMul}.
Let us look at Eq.~\eqref{eq:gpkj_star_def} along with Eq.~\eqref{eq:Fij_def}:

\begin{align}
    \gpk_{j}^{*} &= \prod_{P_{i}\in\mathcal{Q}}F_{i}(j) \nonumber\\
        &= \prod_{P_{i}\in\mathcal{Q}}
                \brackets{\prod_{k=0}^{t} C_{ik}^{j^{k}}}
            \nonumber\\
        &= \prod_{k=0}^{t}
            \brackets{\prod_{P_{i}\in\mathcal{Q}} C_{ik}^{j^{k}}} \nonumber\\
        &= \prod_{k=0}^{t}
            \brackets{\prod_{P_{i}\in\mathcal{Q}} C_{ik}}^{j^{k}}
\end{align}

\noindent
Inside the brackets, we perform $n-1$ \textsc{ECAdd} operations,
and these are performed $t+1$ times.
Then, there are $t$ \textsc{ECMul} operations.
Finally, there are $t$ additional \textsc{ECAdd} operations.
Combining these, there are $t$ calls to \textsc{ECMul},
$nt$ calls to \textsc{ECAdd}, and one call to \textsc{PairingCheck}.
Thus, we see

\begin{equation}
    \text{Cost of Standard Proof} \sim 113000 + 4000n + 100n^{2}.
    \label{eq:cost_standard}
\end{equation}

\noindent
For $n=20$, this corresponds to $223$K gas;
the gas limit is 10M.

We now look at the cost of the group method
and start by determining the cost of the group signature.
The computation consists of forming

\begin{equation}
    \sigma = \prod_{P_{j}\in\mathcal{R}^{*}_{k}} \sigma_{j}^{R_{j}},
\end{equation}

\noindent
where

\begin{equation}
    \mathcal{R}^{*}_{k} = \parens{\mathcal{R}\setminus P_{v}}\cup
        \braces{P_{k}}
\end{equation}

\noindent
for $P_{v}\in\mathcal{R}$ fixed and $P_{k}\in\mathcal{I}$,
the set of participants who incorrectly shared $\gpk_{j}$.
From there, we would call

\begin{equation}
    \textsc{PairingCheck}(\sigma, \bar{h}_{2}, H_{2C}(M), \text{mpk})
\end{equation}

\noindent
to ensure that this is a valid or invalid signature as appropriate.

Each group signature $\sigma$ will require forming $R_{j}$
and computing $\sigma_{j}^{R_{j}}$.
After computing $R_{j}$, we require one \textsc{ECMul} call
to compute $\sigma_{j}^{R_{j}}$.
The expensive part turns out to be forming $R_{j}$,
and we focus on its straightforward computation.
We recall from Eq.~\eqref{eq:Rj_coefs} that

\begin{equation}
    R_{j} = \prod_{\substack{P_{k}\in\mathcal{R} \\ k\ne j}} \frac{k}{k-j}.
\end{equation}

\noindent
Naively, we would need to compute $t$ finite field inversions,
which corresponds to $t$ \textsc{ModExp} calls, to compute
$R_{j}$.
This must be performed $t+1$ times to form $\sigma$.
If we precompute these inverses and include them
in the function call, then this will greatly reduce the overall gas cost.
At the beginning of the call, we check to make sure that the submitted
inverses are valid; if they are valid, we proceed with the
accusation, and if they are invalid, we stop.
See Alg.~\ref{alg:grpsig_malicious} for the complete description.
There are $t+1$ \textsc{ECMul} operations, $t$ \textsc{ECAdd} operations,
and one pairing check per signature check.
Thus, the cost to check one signature is

\begin{equation}
    \text{Cost of Efficient Group Proof} \sim 113000 + 4100n.
    \label{eq:cost_group}
\end{equation}

\noindent
For $n=20$, this corresponds to $195$K gas.
This is sligntly more efficient than computing $\gpk_{j}^{*}$,
but this is not the whole story.

We now take time to make a valid comparison between the
standard accusation (computing $\gpk_{j}$ and group accusation
(computing group signatures).
First, the standard accusation additional gas cost associated
with submitted and verifying public information related to the
encrypted shares and public coefficients;
we do not include an analysis of these costs at this time
and will look at doing that in the future.

By directly comparing Eq.\eqref{eq:cost_standard} and
Eq.~\eqref{eq:cost_group}, it would appear that the group accusation
is more efficient.
Of course, we must remember if there are $m$ malicious participants,
then there are $m$ computations of $\gpk_{j}$,
while there are $m+1$ computations of group signatures.
Furthermore, group signatures have the additional restriction that
there \emph{must} be a honest majority.
Thus, for a valid comparison to be made, we must know the
number of participants as well as the total number of malicious actors.
Tab.~\ref{tab:gpkj_gas_cost} lists the gas cost for select values.
Note that the table does not include the cost associated
with entering the necessary information for the standard accusation;
even so, this is reasonable estimate.

\begin{table}
\centering
\begin{tabular}{|c||r|r||r|r||r|r|}
\hline
  & \multicolumn{2}{|c||}{$n = 10$} & \multicolumn{2}{|c||}{$n = 20$} &
    \multicolumn{2}{|c|}{$n = 30$} \\
\hline
    $m$ & \multicolumn{1}{|c|}{Std} & \multicolumn{1}{|c||}{Grp} &
          \multicolumn{1}{|c|}{Std} & \multicolumn{1}{|c||}{Grp} &
          \multicolumn{1}{|c|}{Std} & \multicolumn{1}{|c||}{Grp} \\
\hline
    1 &  164 &  308 &  236 &  390 &  328 &  472 \\
    2 &  329 &  462 &  472 &  585 &  655 &  708 \\
    3 &  494 &  616 &  708 &  780 &  982 &  944 \\
    4 &  658 &      &  944 &  975 & 1310 & 1180 \\
    5 &  822 &      & 1180 & 1170 & 1638 & 1416 \\
    6 &  987 &      & 1416 & 1365 & 1965 & 1652 \\
    7 & 1152 &      & 1652 &      & 2292 & 1888 \\
\hline
\end{tabular}
\caption[Malicious $\gpk_{j}$ Accusation Gas Cost]{
This is the estimated gas cost comparing the standard and group
$\gpk_{j}$ accusation. The cost is listed in thousands of gas
for the Ethereum Virtual Machine.
Here, $n$ is the number of total participants and $m$ is the number
of malicious actors.
There are no entries in the group method when group accusations
are not possible due to no BFT majority.
}
\label{tab:gpkj_gas_cost}
\end{table}


In all cases, multiple malicious actors could be accused
in a single transaction and still stay well below the 10M gas limit.
Due to the different cost functions, the exact method will depend
on how many validators are dishonest and the total number of participants.
The standard accusation always works even though the cost function
grows more quickly.

\input{algs/gpkj_algs.tex}

